% !TEX encoding = UTF-8 Unicode
%% rubric.tex --- Example of using CurVe.

%% Copyright (C) 2000, 2001, 2002, 2003, 2004, 2005 Didier Verna.

%% Author:        Didier Verna <didier@lrde.epita.fr>
%% Maintainer:    Didier Verna <didier@lrde.epita.fr>
%% Created:       Thu Dec 10 16:04:01 2000
%% Last Revision: Fri Feb  6 17:38:00 2004

%% This file is part of CurVe.

%% CurVe may be distributed and/or modified under the
%% conditions of the LaTeX Project Public License, either version 1.1
%% of this license or (at your option) any later version.
%% The latest version of this license is in
%% http://www.latex-project.org/lppl.txt
%% and version 1.1 or later is part of all distributions of LaTeX
%% version 1999/06/01 or later.

%% CurVe consists of the files listed in the file `README'.
\continuedname{}
\begin{rubric}{%Emploie Saisonnier
}

%----------------------------------------------------------
%   EXPERIENCES PROFESSIONNELLES
%----------------------------------------------------------
\subrubric{Professional Experience}

% 15/07/2014
\entry*[July 2014 - present]
\textbf{Software engineer} at \href{http://www.foretagsplatsen.se/}{\underline{\textsc{Företagsplatsen}}}: front-end developer, UI~\&~UX

\begin{itemize}
\item Front-end development

  JavaScript development, client-side architecture, UX~\&~UI design (Less/CSS,
  SVG~icons).  Unit testing is an important part of the development process, we
  have been using QUnit for JavaScript testing and NUnit on the server-side.

\item Backend development

Backend in \Csharp (with a couchdb database)

\item CI~\&~DevOps

Experience with automated testing, deployment on Microsoft Azure using Ansible.
\end{itemize}

% \mbox{}

% Technologies:
% \begin{AutoMultiColItemize}
% 	\item[- ] JavaScript
% 	\item[- ] Less
% 	\item[- ] Grunt
% 	\item[- ] QUnit
% 	\item[- ] \Csharp
% 	\item[- ] Ansible
% \end{AutoMultiColItemize}

% \mbox{}

% Keywords:
% \begin{AutoMultiColItemize}
% 	\item[- ] Legacy code
% 	\item[- ] Web technologies
% \end{AutoMultiColItemize}

% 01/07/2013 - 31/08/2013
\entry*[July - August 2013]

\textbf{Scientific engineer} at \textsc{Inria} to redesign of the Pharo
Smalltalk IDE

\mbox{}

My position at the \href{http://rmod.inria.fr/}{\underline{RMoD}} team involved
refactoring the legacy Morphic UI framework, implementing a UI-generation
framework in Smalltalk, and writing a full IDE solution for Pharo.

% \mbox{}

% Technologies:
% \begin{AutoMultiColItemize}
% 	\item[- ] Smalltalk
% 	\item[- ] Morphic
% 	\item[- ] Git
% 	\item[- ] Spec
% \end{AutoMultiColItemize}

% \mbox{}

% Keywords:
% \begin{AutoMultiColItemize}
% 	\item[- ] Ergonomics
% 	\item[- ] User experience
% \end{AutoMultiColItemize}

% 01/06/2013 - 30/06/2014
\entry*[June 2013 - June 2014] \textbf{Software engineer consultant} for \href{http://www.foretagsplatsen.se/}{\underline{\textsc{Företagsplatsen}}}

\begin{itemize}
\item Involvement in the development of a new major release of their web
  application: technical migration to a JavaScript-based user interface.
\item Development of a cloud-based document archive application \emph{à la} Dropbox for
  accounting agencies.
\end{itemize}

% 01/06/2013 - 31/08/2013
\entry*[June - August 2013] \textbf{Software engineer} Student program in the \textsc{Google Summer of Code}

\mbox{}

During the \textsc{Google Summer of Code}, I worked on the \emph{Spec UI
  framework}, extracting its code base and uncoupling it from the Morphic UI
legacy framework.

% \begin{itemize}
% 	\item[- ] Work autonomy
% 	\item[- ] Monthly reports
% \end{itemize}

% \entry*[01/06/2013]
% Beginning of my freelance consulting activity.
% \begin{itemize}
% 	\item[- ] Autonomy
% 	\item[- ] Rigor
% \end{itemize}

% 01/03/2013 - 31/03/2013
\entry*[March 2013] \textbf{Young Engineer} at \textsc{Inria}: Development of UI
elements, cleaning of legacy code

\mbox{}

As part of the \href{http://rmod.inria.fr/}{\underline{RMoD}} team, I was in
charge of the implementation of new Morphic widgets, as well as various
refactorings of decades old Morphic code.

% \begin{itemize}
% 	\item[- ] Work autonomy
% 	\item[- ] User interface
% \end{itemize}

% 26/08/2012 - 01/09/2012
\entry*[August 2012] \textbf{Head of the Student volunteer program}:
International Smalltalk conference

\mbox{}

As head of the student volunteers, I was in charge of the coordination of the
students and the well-being of the attendees at the \emph{20th International
  Smalltalk Conference}.

% \begin{itemize}
% 	\item[- ] Team management
% 	\item[- ] Reactivity
% \end{itemize}

% 28/08/2012
\entry*[August 2012] \textbf{First author} of \emph{A Framework for the Specification and Reuse of UIs and their Models}

\mbox{}

Publication of the article \emph{A Framework for the Specification and Reuse of UIs and their Models} by Benjamin \textsc{Van~Ryseghem}, Stéphane \textsc{Ducasse}, and Johan \textsc{Fabry} at IWST~'12.

% 01/05/2012 - 31/08/2012
\entry*[May - August 2012] \textbf{Software engineer} Student program in the \textsc{Google Summer of Code}

\mbox{}

During the Google Summer of Code I had for project to implement the \emph{Traits support in Nautilus},
the new default Pharo IDE I previously wrote.

% 01/07/2012 - 15/08/2012
\entry*[July - August 2012] \textbf{Young engineer} at \textsc{Inria} to refactor and improve widgets

\mbox{}

Employed by the \textsc{Inria} within the \href{http://rmod.inria.fr/}{\underline{RMoD}} team in charge of the improvement of widgets and the development of a widgets generation framework.

%\begin{itemize}
%	\item[- ] User interface
%	\item[- ] Ergonomic
%\end{itemize}

% Summer 2011
\entry*[Summer 2011] \textbf{Software engineer} Student program in the \textsc{SummerTalk}

\mbox{}

The SummerTalk is an \textsc{Esug} (European Smalltalk User Group) equivalent to the Google Summer of Code. My project was to improve the Pharo IDE tools.

%\begin{itemize}
%	\item[- ] Monthly reports
%	\item[- ] Work autonomy
%\end{itemize}

\subrubric{FOSS Projects}

&&This is only few of the FLOSS projects to which I contributed. \\
&& For more, please see my GitHub profile.\\

\mbox{}\\

\entry*[great-things-done] \textbf{Great Things Done} is a keyboard centric \textsc{Gtd} application based on Electron, and implemented in \mbox{Clojure/ClojureScript}. The front-end uses Reagent (an adaptor to React in CLJS), and implements some interesting features like a fully encrypted file-based database, global shortcut with OS X integrations, or Dock icon support. The code is distributed under EPL 1.0.

\entry*[teamwall] \textbf{Teamwall} is designed to help the members of a remote team to feel close to each other. It provides a page with a picture of all the team members, the pictures being refreshed every minute. It allows to know in a glance who is there, who is in pause, or to spot when a possible interesting discussion occurs. The code is distributed under GPL 3.0.

\entry*[ergotron-firmware] \textbf{ergotron} is a hand-made custom keyboard I built from scratch. The firmware is based on the excellent ergodox-firmware by Ben Blazak. The firmrware has been extended to support a lot more keys and LEDs, leading to a new hardware layout. The code is distributed under MIT.

%----------------------------------------------------------
%   COMPETENCES TECHNIQUES
%----------------------------------------------------------
\subrubric{Technical skills}
\entry*[Programming] \textbf{JavaScript}, Clojure, Smalltalk, \textsc{Java}, C, \textsc{SQL}, \textsc{Prolog}, \textsc{Caml}, \textsc{Cobol}
\entry*[Software] \textbf{Git}, \textbf{WebStorm}, TunnelBlick, Slack
\entry*[System] \textsc{*nix}, Windows, AS400, shell
\entry*[Agile] \textbf{Remote working}, Pair-programming, Scrum, TDD, Pomodoro, GTD
%\entry*[Web] J2EE, HTML/CSS, Flash (autodidacte).

%----------------------------------------------------------
%   FORMATION
%----------------------------------------------------------
\subrubric{Education}
\entry*[2012 - 2013]
1\textup{st} year of Master Informatique at the Université des Sciences et Technologies de Lille.

\entry*[2011 - 2012]
3\textup{rd} year of Licence Informatique (Bachelor degree) at the Université des Sciences et Technologies de Lille.

\entry*[2009 - 2011]
DUT Informatique (a 2 year technical degree in Computer Science)  at the IUT A de Lille (major).
%\begin{itemize}
% \item[] Multidisciplinary formation:
% \item[] \begin{itemize}
%        \item[- ] Programming languages
%	\item[- ] Communication
%%	\item[- ] Connaissance de l'entreprise (SSII) : gestion d'entreprise, Žconomie, droit
%       \end{itemize}
%\end{itemize}

\entry*[2005 - 2009]
Classe Préparatoire aux Grandes Écoles (Higher School Preparatory Classes), Mathematics and Physics with Computer Science as option at Roosevelt and Clémenceau at Reims then 3\textup{rd} year of Licence de Mathématiques Pures et Appliquées (Bachelor degree) at the Université des Sciences et Technologies de Lille.
%\begin{itemize}
% \item[- ]  Stress Management
%  \item[- ]  Ability to synthesize
%%  \item[- ]  Autonomie.
%  \item[- ]  Rigour
%\end{itemize}
%\entry*[2007 - 2009]
%3\textup{me} annŽe de Licence de MathŽmatiques Pures et AppliquŽes ˆ l'UniversitŽ des Sciences et Technologies de Lille.
%\begin{itemize}
%  \item[- ]  Esprit de synthse
%%  \item[- ]  Autonomie.
%  \item[- ]  Rigueur

%\end{itemize}
%\entry*[2005 - 2007]
%Classe PrŽparatoire aux Grandes ƒcoles, MathŽmatiques-Physique option Informatique aux lycŽes Roosevelt et ClŽmenceau de Reims.
%\begin{itemize}
%  \item[- ]  Gestion du stress
%  \item[- ]  CompŽtitivitŽ
%  \item[- ]  QualitŽ de communication
%\end{itemize}

\entry*[2005]
Graduation of a High School Diploma "Scientifique option Sciences de l'Ingénieur, spécialité Mathématiques" (Scientific highschool diploma, with an engineering sciences option and a Mathematic speciality) at Lycée Joliot-Curie at Romilly sur Seine.


%\entry*[ƒtŽ 2011]
%Participation au SummerTalk organisŽ par \textsc{Esug} (European Smalltalk User Group) pour le projet \emph{Tools improvement}.
%\begin{itemize}
%	\item[- ] ƒcriture de compte-rendus mensuels
%	\item[- ] Autonomie de travail
%\end{itemize}

%\entry*[01/05/2011 - 30/06/2011]
%EmployŽ ˆ l'\textsc{Inria}, en charge de l'amŽlioration des widgets et du dŽveloppement de nouveaux outils d'Ždition de code.
%\begin{itemize}
%	\item[- ] Interface utilisateur
%	\item[- ] Ergonomie
%\end{itemize}

%\entry*[01/11/2010 - 21/04/2011\\01/04/2012 - 31/06/2012]
%Stagiaire ˆ l'\textsc{Inria}, en charge du projet \textbf{Hazelnut} (crŽation dynamique de noyaux dans un langage rŽflexif) au sein de l'Žquipe \textsc{\href{http://rmod.inria.fr/}{\underline{RMoD}}}.
%\begin{itemize}
%	\item[- ] Programmation orientŽe objet
%	\item[- ] RŽdaction de rapports techniques
%	\item[- ] Gestion autonome de projets
%\end{itemize}



%\entry*[2010]
%Projet tuteurŽ : Ajout d'un plugin ˆ Pharo, une IDE pour \textsc{SmallTalk}
%\begin{itemize}
%  \item[- ]  Sens de l'initiative.
%  \item[- ]  Travail d'Žquipe.
%  \item[- ]  Respect des dŽlais.
%  \item[- ]  Gestion d'un projet, de l'analyse au recettage.
%\end{itemize}
%\entry*[2009]
%EmployŽ polyvalent ˆ la SARL \textsc{What's Up} (salle d'escalade) ˆ Villeneuve d'Ascq.
%\begin{itemize}
%  \item[- ]  Polyvalence
%  \item[- ]  FacultŽ d'adaptation
%\end{itemize}

%\entry*[2006 - 2008]
%Libraire durant un mois, puis temps complet partagŽ Service DŽcoration/H™te de Caisse durant un mois, puis Service DŽcoration durant un mois au centre \textsc{E.Leclerc} Sodirom, Romilly-sur-Seine (emploi saisonnier).
%\begin{itemize}
%  \item[- ]  Respect de chartes internes (charte graphique, quantitŽs et dimensions des impressions).
%  \item[- ]  CrŽativitŽ.
%\end{itemize}
%----------------------------------------------------------
%   CENTRES D'INTERETS
%----------------------------------------------------------
\subrubric{Hobbies and interests}
\entry*[Sport]
Rock climbing during 10 years, including 5 years at a competition level. Supervision of children (between 10 years old and 15 years old) during 2 years.

\entry*[Cooking] I love to cook for my family and friends, and have a nice moment all together.
%\entry*[Drawing]
%Practice (charcoal, pastel, numeric) and comic book lover, especially books by (\textsc{Bilal} and \textsc{Ledroit}).

% \entry*[Films] Eclectic tastes, lover of \textsc{Brian de Palma}'s work.

\entry*[Movies] I love to watch movies. All kind of movies. I can't resists movies that are so bad they're good!

\entry*[Mechanical keyboards] I spent a lot of time trying to find the ultimate keyboard. Until I made mine myself.

\end{rubric}

%% rubric.tex ends here.

%%% Local Variables:
%%% mode: latex
%%% TeX-master: "cv"
%%% End:
