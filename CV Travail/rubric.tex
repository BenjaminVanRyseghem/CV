% !TEX encoding = UTF-8 Unicode
%% rubric.tex --- Example of using CurVe.

%% Copyright (C) 2000, 2001, 2002, 2003, 2004, 2005 Didier Verna.

%% Author:        Didier Verna <didier@lrde.epita.fr>
%% Maintainer:    Didier Verna <didier@lrde.epita.fr>
%% Created:       Thu Dec 10 16:04:01 2000
%% Last Revision: Fri Feb  6 17:38:00 2004

%% This file is part of CurVe.

%% CurVe may be distributed and/or modified under the
%% conditions of the LaTeX Project Public License, either version 1.1
%% of this license or (at your option) any later version.
%% The latest version of this license is in
%% http://www.latex-project.org/lppl.txt
%% and version 1.1 or later is part of all distributions of LaTeX
%% version 1999/06/01 or later.

%% CurVe consists of the files listed in the file `README'.
\continuedname{}
\begin{rubric}{%Emploie Saisonnier
}

%----------------------------------------------------------
%   EXPERIENCES PROFESSIONNELLES
%----------------------------------------------------------
\subrubric{Expériences Professionnelles}

\entry*[15/07/2014]
Ingénieur logicielle au sein de \href{http://www.foretagsplatsen.se/}{\underline{Företagsplatsen}}.
\begin{itemize}
	\item[- ] Legacy code
	\item[- ] Technologies Web
\end{itemize}

\entry*[01/07/2013 - 31/08/2013]
Ingénieur scientifique à Inria au sein de l'équipe \href{http://rmod.inria.fr/}{\underline{RMoD}} en charge de la refonte complète de l'environnement de développement intégrée de Pharo.
\begin{itemize}
	\item[- ] Ergonomie
	\item[- ] Expérience utilisateur
\end{itemize}

\entry*[01/06/2013 - 30/06/2014]
Consultant au sein de l'équipe de développement de \href{http://www.foretagsplatsen.se/}{\underline{Företagsplatsen}} (entreprise suédoise d'aide à la prise de décision) afin de finaliser la migration technologique de leurs applications.
\begin{itemize}
	\item[- ] Télé-travail
	\item[- ] Communication
\end{itemize}

\entry*[01/06/2013 - 31/08/2013]
Participation au Google Summer of Code pour le projet \emph{Spec framework independency}.
\begin{itemize}
	\item[- ] Autonomie de travail
	\item[- ] Rapports mensuels
\end{itemize}

\entry*[01/06/2013]
Début de l'activité de consulting en autoentreprenariat.
\begin{itemize}
	\item[- ] Autonomie
	\item[- ] Rigueur
\end{itemize}

\entry*[01/03/2013 - 31/03/2013]
ASSET scientifique à Inria au sein de l'équipe \href{http://rmod.inria.fr/}{\underline{RMoD}} en charge de l'implémentation de nouveaux widgets Morphic.
\begin{itemize}
	\item[- ] Autonomie de travail
	\item[- ] Interface utlisateur
\end{itemize}


\entry*[26/08/2012 - 01/09/2012]
En charge du programme d'étudiants volontaires lors de la conférence \emph{20th International Smalltalk Conference}.
\begin{itemize}
	\item[- ] Gestion d'équipe
	\item[- ] Réactivité
\end{itemize}


\entry*[28/08/2012]
Publication de l'article \emph{A Framework for the Specification and Reuse of UIs and their Models} par Benjamin \bsc{Van Ryseghem}, Stéphane \bsc{Ducasse} et Johan \bsc{Fabry} à IWST~'12.
\begin{itemize}
	\item[- ] Rigueur scientifique
	\item[- ] Écriture de papier scientifique
\end{itemize}


\entry*[01/05/2012 - 31/08/2012]
Participation au Google Summer of Code pour le projet \emph{Traits support in Nautilus}.
\begin{itemize}
	\item[- ] Autonomie de travail
	\item[- ] Rapports mensuels
\end{itemize}

\entry*[01/07/2012 - 15/08/2012]
Employé à l'\bsc{Inria}, en charge de l'amélioration des widgets et du développement d'un framework de génération de widgets
\begin{itemize}
	\item[- ] Interface utilisateur
	\item[- ] Ergonomie
\end{itemize}

\entry*[Été 2011]
Participation au SummerTalk organisé par \bsc{Esug} (European Smalltalk User Group) pour le projet \emph{Tools improvement}.
\begin{itemize}
	\item[- ] Écriture de compte-rendus mensuels
	\item[- ] Autonomie de travail
\end{itemize}


%----------------------------------------------------------
%   COMPETENCES TECHNIQUES
%----------------------------------------------------------
\subrubric{Compétences Techniques}
\entry*[Programmation] \textbf{JavaScript}, Clojure, Smalltalk, \bsc{Java}, C\#, C/C++, \bsc{Sql}, \bsc{Prolog}, \bsc{Caml}, \bsc{Cobol}.
\entry*[Système] \bsc{Unix}, OSX, Windows, AS400, shell.
\entry*[Analyse] UML, Merise.
\entry*[Web] HTML 5, CSS 3.

%----------------------------------------------------------
%   FORMATION
%----------------------------------------------------------
\subrubric{Formation}
\entry*[2012 - 2013]
1\up{ère} année de Master Informatique à l'Université des Sciences et Technologies de Lille.

\entry*[2011 - 2012]
3\up{ème} année de Licence Informatique à l'Université des Sciences et Technologies de Lille.

\entry*[2009 - 2011]
DUT Informatique à l'IUT A de Lille (major).
\begin{itemize}
 \item[] Formation pluridisciplinaire à dominante pratique :
 \item[] \begin{itemize}
        \item[- ] Langages de programmation
	\item[- ] Communication
%	\item[- ] Connaissance de l'entreprise (SSII) : gestion d'entreprise, économie, droit
	
       \end{itemize}
\end{itemize}

\entry*[2005 - 2009]
Classe Préparatoire aux Grandes Écoles, Mathématiques-Physique option Informatique aux lycées Roosevelt et Clémenceau de Reims puis 3\up{ème} année de Licence de Mathématiques Pures et Appliquées à l'Université des Sciences et Technologies de Lille.
\begin{itemize}
 \item[- ]  Gestion du stress
  \item[- ]  Esprit de synthèse
%  \item[- ]  Autonomie.
  \item[- ]  Rigueur
\end{itemize}
%\entry*[2007 - 2009]
%3\up{ème} année de Licence de Mathématiques Pures et Appliquées à l'Université des Sciences et Technologies de Lille.
%\begin{itemize}
%  \item[- ]  Esprit de synthèse
%%  \item[- ]  Autonomie.
%  \item[- ]  Rigueur

%\end{itemize}
%\entry*[2005 - 2007]
%Classe Préparatoire aux Grandes Écoles, Mathématiques-Physique option Informatique aux lycées Roosevelt et Clémenceau de Reims.
%\begin{itemize}
%  \item[- ]  Gestion du stress
%  \item[- ]  Compétitivité
%  \item[- ]  Qualité de communication
%\end{itemize}

\entry*[2005]
Obtention d'un Baccalauréat Scientifique option Sciences de l'Ingénieur, spécialité Mathématiques au Lycée Joliot-Curie de Romilly sur Seine.

\entry*[Anglais]
Maitrise.


%\entry*[Été 2011]
%Participation au SummerTalk organisé par \bsc{Esug} (European Smalltalk User Group) pour le projet \emph{Tools improvement}.
%\begin{itemize}
%	\item[- ] Écriture de compte-rendus mensuels
%	\item[- ] Autonomie de travail
%\end{itemize}

%\entry*[01/05/2011 - 30/06/2011]
%Employé à l'\bsc{Inria}, en charge de l'amélioration des widgets et du développement de nouveaux outils d'édition de code.
%\begin{itemize}
%	\item[- ] Interface utilisateur
%	\item[- ] Ergonomie
%\end{itemize}

%\entry*[01/11/2010 - 21/04/2011\\01/04/2012 - 31/06/2012]
%Stagiaire à l'\bsc{Inria}, en charge du projet \textbf{Hazelnut} (création dynamique de noyaux dans un langage réflexif) au sein de l'équipe \bsc{\href{http://rmod.inria.fr/}{\underline{RMoD}}}.
%\begin{itemize}
%	\item[- ] Programmation orientée objet
%	\item[- ] Rédaction de rapports techniques
%	\item[- ] Gestion autonome de projets
%\end{itemize}



%\entry*[2010]
%Projet tuteuré : Ajout d'un plugin à Pharo, une IDE pour \bsc{SmallTalk}
%\begin{itemize}
%  \item[- ]  Sens de l'initiative.
%  \item[- ]  Travail d'équipe.
%  \item[- ]  Respect des délais.
%  \item[- ]  Gestion d'un projet, de l'analyse au recettage.
%\end{itemize}
%\entry*[2009]
%Employé polyvalent à la SARL \bsc{What's Up} (salle d'escalade) à Villeneuve d'Ascq.
%\begin{itemize}
%  \item[- ]  Polyvalence
%  \item[- ]  Faculté d'adaptation
%\end{itemize}

%\entry*[2006 - 2008]
%Libraire durant un mois, puis temps complet partagé Service Décoration/H™te de Caisse durant un mois, puis Service Décoration durant un mois au centre \bsc{E.Leclerc} Sodirom, Romilly-sur-Seine (emploi saisonnier).
%\begin{itemize}
%  \item[- ]  Respect de chartes internes (charte graphique, quantités et dimensions des impressions).
%  \item[- ]  Créativité.
%\end{itemize}
%----------------------------------------------------------
%   CENTRES D'INTERETS
%----------------------------------------------------------
\subrubric{Centres d'intérêt}
\entry*[Sport]
Escalade à un niveau inter-régional durant 5 ans. Encadrement d'enfants (de 10 à 15 ans) pendant 2 ans.
\entry*[Dessin]
Pratique (fusain, pastel, numérique) et bédéphile (\bsc{Bilal} et \bsc{Ledroit} particulièrement).
 \entry*[Cinéma] Goûts éclectiques, amateur de \bsc{Brian de Palma}.

\end{rubric}

%% rubric.tex ends here.

%%% Local Variables:
%%% mode: latex
%%% TeX-master: "cv"
%%% End:
