% !TEX encoding = UTF-8 Unicode
%% rubric.tex --- Example of using CurVe.

%% Copyright (C) 2000, 2001, 2002, 2003, 2004, 2005 Didier Verna.

%% Author:        Didier Verna <didier@lrde.epita.fr>
%% Maintainer:    Didier Verna <didier@lrde.epita.fr>
%% Created:       Thu Dec 10 16:04:01 2000
%% Last Revision: Fri Feb  6 17:38:00 2004

%% This file is part of CurVe.

%% CurVe may be distributed and/or modified under the
%% conditions of the LaTeX Project Public License, either version 1.1
%% of this license or (at your option) any later version.
%% The latest version of this license is in
%% http://www.latex-project.org/lppl.txt
%% and version 1.1 or later is part of all distributions of LaTeX
%% version 1999/06/01 or later.

%% CurVe consists of the files listed in the file `README'.
\continuedname{}

\newcommand{\link}[2]{\href{#2}{\underline{\smash{#1}}}}

\setlength{\leftmargini}{1em}

\begin{rubric}{
}

%----------------------------------------------------------
%   EXPERIENCES PROFESSIONNELLES
%----------------------------------------------------------
\subrubric{Professional Experience}

\entry*[March 2019 - Today]
\textbf{Lead full stack engineer} at \link{\textsc{Cyberzen}}{https://cyberzen.com}: Cybersecurity apps
\begin{itemize}

\item Full stack development

Development of in-house applications to audit companies security policies built upon a full Javascript stack. We use Express server-side, and React front-side to build modern applications.

\item Digital transition

Implementation of tools to help companies to transition smoothly into the numeric world. We design solutions to accelerate companies growth and to convert their activities to use today's technologies and capabilities.

\item Cybersecurity Android app

Development of an Android app using Kotlin to scan credit card and display informations stored on a contactless payment card.

\item Remote

Setup of a remote-friendly environment as the first remote-employee of the company.

\end{itemize}

\entry*[October 2017 - February 2019]
\textbf{Senior front-end engineer} at \link{\textsc{Weezevent}}{https://weezevent.com/fr/}: seated ticketing solution

\begin{itemize}
\item Seated ticketing

Development of a seated ticketing solution in multiple layers from raw map drawing using D3 to 
a complete sales workflow for final users. Attention was given to the design to allow multiple rendering
solutions (a in-house solution in 2d as well as \link{\textsc{Pacifa}}{pacifa3d.com/} for 3d rendering),
but also to allow several part of the application to plug onto the solution to add their own interactions.

\item Front-end development

In charge of improving the overall javascript code and practices as well as bringing the latest of the front-end ecosystem
to improve the quality of the code and the skills of other devs when it comes to Javascript. We use React, Redux (we implemented
a switchless layer as I personally don't like strings comparison over OOP message sending), and the react eco-system (jest, create-react-app, etc).

\item Remote

Setup of a remote-friendly environment as the first remote-employee of the company. Teaching of good-practices experienced over years:
good communication process, async meeting, useful scripts, etc.

\end{itemize}

% 15/07/2014
\entry*[July 2014 - June 2017]
\textbf{Software engineer} at \link{\textsc{Företagsplatsen}}{http://www.foretagsplatsen.se/}: full-stack engineer, UX~\&~UI

\begin{itemize}
\item Front-end development

  JavaScript development (ES5/ES6), client-side architecture, UX~\&~UI design (Less/CSS,
  SVG~icons).  Unit testing is an important part of the development process, we
  are using Jasmine for JavaScript testing (through Karma).

\item Backend development

Backend in \Csharp, using the MVP framework for REST API handling. The server
fetches the data from a CouchDB database. We use NUnit on the server-side for unit testing.

\item CI~\&~DevOps

Experience with automated testing via TeamCity.
Automatic deployment on Microsoft Azure using Ansible.
\end{itemize}

% \medskip

% Technologies:
% \begin{AutoMultiColItemize}
% 	\item[- ] JavaScript
% 	\item[- ] Less
% 	\item[- ] Grunt
% 	\item[- ] QUnit
% 	\item[- ] \Csharp
% 	\item[- ] Ansible
% \end{AutoMultiColItemize}

% \medskip

% Keywords:
% \begin{AutoMultiColItemize}
% 	\item[- ] Legacy code
% 	\item[- ] Web technologies
% \end{AutoMultiColItemize}

% 01/07/2013 - 31/08/2013
\entry*[July - August 2013]

\textbf{Scientific engineer} at \textsc{Inria} (\link{RMoD}{http://rmod.inria.fr/}): redesign of the Pharo
Smalltalk IDE

\begin{itemize}
    \item Responsible for the refactoring of the legacy "Morphic UI" framework.
    \item Implementation of \link{Spec}{https://benjamin.vanryseghem.com/projects/spec/}: a UI-generation framework.
    \item Development of a fully-featured IDE solution for \link{Pharo}{http://pharo.org/}.
\end{itemize}

% \medskip

% Technologies:
% \begin{AutoMultiColItemize}
% 	\item[- ] Smalltalk
% 	\item[- ] Morphic
% 	\item[- ] Git
% 	\item[- ] Spec
% \end{AutoMultiColItemize}

% \medskip

% Keywords:
% \begin{AutoMultiColItemize}
% 	\item[- ] Ergonomics
% 	\item[- ] User experience
% \end{AutoMultiColItemize}

% 01/06/2013 - 30/06/2014
\entry*[June 2013 - June 2014] \textbf{Software engineer consultant} at \link{\textsc{Företagsplatsen}}{http://www.foretagsplatsen.se/}

\begin{itemize}
\item Development of a new major release of the web
  application: technical migration from a template-based server-side application to a component-based JavaScript single-page application.
\item Development of a cloud-based document archive application \emph{à la} Dropbox for
  accounting agencies.
\end{itemize}

% 01/06/2013 - 31/08/2013
\entry*[June - August 2013] \textbf{Software engineer} Student program of the \textsc{Google Summer of Code}

\medskip

Improvement of \link{Spec}{https://benjamin.vanryseghem.com/projects/spec/}: decoupling the models from the UI framework for better extensibility.

% \begin{itemize}
% 	\item[- ] Work autonomy
% 	\item[- ] Monthly reports
% \end{itemize}

% \entry*[01/06/2013]
% Beginning of my freelance consulting activity.
% \begin{itemize}
% 	\item[- ] Autonomy
% 	\item[- ] Rigor
% \end{itemize}

% 01/03/2013 - 31/03/2013
\entry*[March 2013] \textbf{Young Engineer} at \textsc{Inria} (\link{RMoD}{http://rmod.inria.fr/}): UI framework development

\begin{itemize}
    \item Implementation of new UI widgets (in Morphic).
    \item Refactoring of the legacy UI codebase.
\end{itemize}

% \begin{itemize}
% 	\item[- ] Work autonomy
% 	\item[- ] User interface
% \end{itemize}

% 26/08/2012 - 01/09/2012
\entry*[August 2012] \textbf{Head of the Student volunteer program} at International Smalltalk conference

\medskip

Coordination of the students and the well-being of the attendees at the \emph{20th International Smalltalk Conference}.

% \begin{itemize}
% 	\item[- ] Team management
% 	\item[- ] Reactivity
% \end{itemize}

% 28/08/2012
\entry*[August 2012] \textbf{First author} of \emph{A Framework for the Specification and Reuse of UIs and their Models}

\medskip

Publication of the article \emph{A Framework for the Specification and Reuse of UIs and their Models} by Benjamin \textsc{Van~Ryseghem}, Stéphane \textsc{Ducasse}, and Johan \textsc{Fabry} at IWST~'12.

% 01/05/2012 - 31/08/2012
\entry*[May - August 2012] \textbf{Software engineer} Student program of the \textsc{Google Summer of Code}

\medskip

Implementation the \emph{Traits support in Nautilus}, the new default Pharo IDE I previously developed.

% 01/07/2012 - 15/08/2012
\entry*[July - August 2012] \textbf{Young engineer} at \textsc{Inria} (\link{RMoD}{http://rmod.inria.fr/}): Refactor and improve widgets

\medskip

Improvement of widgets and the development of a widgets generation framework.

%\begin{itemize}
%	\item[- ] User interface
%	\item[- ] Ergonomic
%\end{itemize}

% Summer 2011
\entry*[Summer 2011] \textbf{Software engineer} Student program of the \textsc{SummerTalk}

\medskip

Improvement of the \link{Pharo}{http://pharo.org/} IDE toolset. The SummerTalk is an \textsc{Esug} (European Smalltalk User Group) equivalent to the Google Summer of Code.

%\begin{itemize}
%	\item[- ] Monthly reports
%	\item[- ] Work autonomy
%\end{itemize}

\subrubric{FLOSS Projects}

Pharo&&As a core maintainer of Pharo (an open source Smalltalk implementation), most of my Smalltalk projects have been integrated in the distribution. My other Smalltalk projects can be found on \link{Smalltalkhub}{http://smalltalkhub.com/\#\!/~BenjaminVanRyseghem}.\\\\
&&Following is a selection of FLOSS projects of which I am the author. \\
&& Please see my \link{GitHub profile}{https://github.com/BenjaminVanRyseghem} for a more complete list.\\

\medskip\\

\entry*[git-linter] \textbf{git-linter} is a command line tool and a docker-based GitHub/Gitlab CI integration that lint git commit messages using project-defined rules. The code is distributed under the GPL 3.0 license.

\entry*[SandGlass] \textbf{SandGlass} is an electron-based systray app used to track my working time. It provides a CLI for workflow integration and D3 based histograms of time per projects. The code is distributed under the GPL 3.0 license.

\entry*[great-things-done] \textbf{Great Things Done} is a keyboard-centric GTD application based on Electron, and implemented in \mbox{Clojure/ClojureScript}. The front-end uses Reagent (an adaptor to React in CLJS), and implements some interesting features like a fully encrypted file-based database, global shortcut with OS X integrations, or Dock icon support. The code is distributed under the EPL 1.0 license.

% \entry*[teamwall] \textbf{Teamwall} is designed to help the members of a remote team to feel close to each other. It provides a page with a picture of all the team members, the pictures being refreshed every minute. It allows to know in a glance who is there, who is in pause, or to spot when a possible interesting discussion occurs. The code is distributed under the GPL 3.0 license.

\entry*[ergotron-firmware] \textbf{ergotron} is a hand-made custom keyboard I built from scratch. The firmware is based on the excellent ergodox-firmware by Ben Blazak. The firmrware has been extended to support a lot more keys and LEDs, leading to a new hardware layout. The code is distributed under the MIT license.

%----------------------------------------------------------
%   COMPETENCES TECHNIQUES
%----------------------------------------------------------
\subrubric{Technical skills}

\entry*[Agile] \textbf{Remote working}, Pair-programming, TDD, Getting Things Done, Scrum, Pomodoro
\entry*[Programming] \textbf{JavaScript}, Clojure, Smalltalk, \textsc{Java}, C, \textsc{SQL}, \textsc{Prolog}, \textsc{Caml}, \textsc{Cobol}
\entry*[Software] \textbf{Git}, \textbf{WebStorm}, \textbf{GitHub}, Gitlab, Slack, Upsource, TunnelBlick
\entry*[Continuous Integration] Travis CI, TeamCity, Jenkins
\entry*[System] \textsc{GNU Linux/UNIX}, OS X, Microsoft Windows, AS400
%\entry*[Web] J2EE, HTML/CSS, Flash (autodidacte).

%----------------------------------------------------------
%   FORMATION
%----------------------------------------------------------
\subrubric{Education}
\entry*[2012 - 2013]
1\textup{st} year of Master Informatique (Master in Computer Science) at the Université des Sciences et Technologies de Lille.

\entry*[2011 - 2012]
3\textup{rd} year of Licence Informatique (Bachelor degree in Computer science) at the Université des Sciences et Technologies de Lille.

\entry*[2009 - 2011]
DUT Informatique (a 2 year technical degree in Computer Science)  at the IUT A de Lille (major).
%\begin{itemize}
% \item[] Multidisciplinary formation:
% \item[] \begin{itemize}
%        \item[- ] Programming languages
%	\item[- ] Communication
%%	\item[- ] Connaissance de l'entreprise (SSII) : gestion d'entreprise, économie, droit
%       \end{itemize}
%\end{itemize}

\entry*[2005 - 2009]
Classe Préparatoire aux Grandes Écoles (Higher School Preparatory Classes), Mathematics and Physics with Computer Science as option at Roosevelt and Clémenceau at Reims then 3\textup{rd} year of Licence de Mathématiques Pures et Appliquées (Bachelor degree) at the Université des Sciences et Technologies de Lille.
%\begin{itemize}
% \item[- ]  Stress Management
%  \item[- ]  Ability to synthesize
%%  \item[- ]  Autonomie.
%  \item[- ]  Rigour
%\end{itemize}
%\entry*[2007 - 2009]
%3\textup{me} année de Licence de Mathématiques Pures et Appliquées à l'Université des Sciences et Technologies de Lille.
%\begin{itemize}
%  \item[- ]  Esprit de synthse
%%  \item[- ]  Autonomie.
%  \item[- ]  Rigueur

%\end{itemize}
%\entry*[2005 - 2007]
%Classe Préparatoire aux Grandes Écoles, Mathématiques-Physique option Informatique aux lycées Roosevelt et Clémenceau de Reims.
%\begin{itemize}
%  \item[- ]  Gestion du stress
%  \item[- ]  Compétitivité
%  \item[- ]  Qualité de communication
%\end{itemize}

\entry*[2005]
Graduation of a High School Diploma "Scientifique option Sciences de l'Ingénieur, spécialité Mathématiques" (Scientific highschool diploma, with an engineering sciences option and a Mathematic speciality) at Lycée Joliot-Curie at Romilly sur Seine.


%\entry*[Été 2011]
%Participation au SummerTalk organisé par \textsc{Esug} (European Smalltalk User Group) pour le projet \emph{Tools improvement}.
%\begin{itemize}
%	\item[- ] Écriture de compte-rendus mensuels
%	\item[- ] Autonomie de travail
%\end{itemize}

%\entry*[01/05/2011 - 30/06/2011]
%Employé à l'\textsc{Inria}, en charge de l'amélioration des widgets et du développement de nouveaux outils d'édition de code.
%\begin{itemize}
%	\item[- ] Interface utilisateur
%	\item[- ] Ergonomie
%\end{itemize}

%\entry*[01/11/2010 - 21/04/2011\\01/04/2012 - 31/06/2012]
%Stagiaire à l'\textsc{Inria}, en charge du projet \textbf{Hazelnut} (création dynamique de noyaux dans un langage réflexif) au sein de l'équipe \textsc{\href{http://rmod.inria.fr/}{\underline{RMoD}}}.
%\begin{itemize}
%	\item[- ] Programmation orientée objet
%	\item[- ] Rédaction de rapports techniques
%	\item[- ] Gestion autonome de projets
%\end{itemize}



%\entry*[2010]
%Projet tuteuré : Ajout d'un plugin à Pharo, une IDE pour \textsc{SmallTalk}
%\begin{itemize}
%  \item[- ]  Sens de l'initiative.
%  \item[- ]  Travail d'équipe.
%  \item[- ]  Respect des délais.
%  \item[- ]  Gestion d'un projet, de l'analyse au recettage.
%\end{itemize}
%\entry*[2009]
%Employé polyvalent à la SARL \textsc{What's Up} (salle d'escalade) à Villeneuve d'Ascq.
%\begin{itemize}
%  \item[- ]  Polyvalence
%  \item[- ]  Faculté d'adaptation
%\end{itemize}

%\entry*[2006 - 2008]
%Libraire durant un mois, puis temps complet partagé Service Décoration/H™te de Caisse durant un mois, puis Service Décoration durant un mois au centre \textsc{E.Leclerc} Sodirom, Romilly-sur-Seine (emploi saisonnier).
%\begin{itemize}
%  \item[- ]  Respect de chartes internes (charte graphique, quantités et dimensions des impressions).
%  \item[- ]  Créativité.
%\end{itemize}
%----------------------------------------------------------
%   CENTRES D'INTERETS
%----------------------------------------------------------
\subrubric{Hobbies and interests}

\entry*[Sport] Rock climbing during 10 years, including 5 years at a competition level. Supervision of children (between 10 years old and 15 years old) during 2 years.

\entry*[Cooking] I love to cook for my family and friends, and have a nice moment all together.
%\entry*[Drawing]
%Practice (charcoal, pastel, numeric) and comic book lover, especially books by (\textsc{Bilal} and \textsc{Ledroit}).

% \entry*[Films] Eclectic tastes, lover of \textsc{Brian de Palma}'s work.

\entry*[Movies] I love to watch movies, all kind of movies. I can't resist movies that are so bad they're good!

\entry*[Mechanical keyboards] I spent a lot of time trying to find the ultimate keyboard. Until I made one myself.

\subrubric{Languages}

\entry*[French] Native language.

\entry*[English] Primary language at work since 2008, both written and spoken.

\end{rubric}

%% rubric.tex ends here.

%%% Local Variables:
%%% mode: latex
%%% TeX-master: "cv"
%%% End:
